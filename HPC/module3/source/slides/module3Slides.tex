%	Use the standard preamble for Beamer slides of all 
%		statsTeachR modules
%		(\input, not \include, as \include can't access 
%		things in higher-level directories since it needs 
%		write permission there, which it doesn't have; and 
%		in some settings the preamble may be in a higher-level
%		directory than the source file.)
%	This path assumes the preamble is in the parent directory,
%		modify this if that changes.
%	************************************************
%	**	LaTeX preamble to be used with all 
%	**	statsTeachR labs/handouts.
%
%	Author: Eric A. Cohen
%	Last modified: 22 August 2013
%	************************************************

\documentclass[table]{beamer}

%	Set theme (a nice plain one)
\usetheme{Malmoe}

%	Use named colors, set main color of theme
%		to match Web site color:
\definecolor{MainColor}{RGB}{10, 74, 109}
\colorlet{MainColorMedium}{MainColor!50}
\colorlet{MainColorLight}{MainColor!20}
\usecolortheme[named=MainColor]{structure} 

%	For tables
%[dvipsnames] [table]
\usepackage{xcolor}
\usepackage{tabu}	% Even fancier than tabulary
\usepackage{multirow}

%	Just for the degree symbol
\usepackage{textcomp}

%	Get rid of footline (page, author, etc. on each slide)
\setbeamertemplate{footline}{}
%	Get rid of navigation buttons
\setbeamertemplate{navigation symbols}{}

%	Make footnotes not ugly
\usepackage{hanging}
\setbeamertemplate{footnote}{\raggedright\hangpara{1em}{1}\makebox[1em][l]{\insertfootnotemark}\footnotesize\insertfootnotetext\par}

%	Text style for code snippets inline in text:
\newcommand{\codeInline}[1]{\texttt{#1}}

%	Text style for emphasis stronger than \emph:
%		(Note, this doesn't toggle the way \emph does.
%			(Note, this can be done, didn't seem worth the trouble.))
\newcommand{\strong}[1]{{\bfseries{#1}}}


%	******	Define title page	**********************
\setbeamertemplate{title page}{
	{\color{MainColor}
	% There must be a better way than this -vspace at
	%	 the top and bottom of the page to reduce the 
	%	 bottom margin, but I can't find one that works.
	\vspace{-6em}

	% Go to a lot of trouble to get the title in a
	%	nice box, since customizing a beamer block
	%	does not entirely work here (I don't know why)
	\newlength{\titleBoxWidth}
	\setlength{\titleBoxWidth}{\textwidth}
	\addtolength{\titleBoxWidth}{-2.0em}
	\setlength{\fboxsep}{1.0em}
	\setlength{\fboxrule}{0pt}
	\fcolorbox{MainColor!25}{MainColor!25}{
		\parbox{\titleBoxWidth}{
			\raggedright
			\LARGE\textbf{\inserttitle}
		}	% end parbox
	}	% end fcolorbox

	\vfill
	\small{Author: \insertauthor}
	\vspace{\baselineskip}

	\small{\Course}

	\small{\Instructor}
	\vspace{\baselineskip}

	\small{\emph{This material is part of the \strong{statsTeachR} project}}

	\vspace{0.33\baselineskip}\scriptsize{\emph{\LicenseText}}





		\vspace{-15em}

	}	% end color
	\clearpage
}	% end define title page




%	The following variables are assumed by the standard preamble:
%	Global variable containing module name:
\title{Cluster Computing}
%	Global variable containing module shortname:
%		(Currently unused, may be used in future.)
\newcommand{\ModuleShortname}{shortName}
%	Global variable containing author name:
\author{Nicholas G Reich, Andrea S Foulkes, and Gregory J Matthews}
%	Global variable containing text of license terms:
\newcommand{\LicenseText}{Made available under the Creative Commons Attribution-ShareAlike 3.0 Unported License: http://creativecommons.org/licenses/by-sa/3.0/deed.en\textunderscore US }
%	Instructor: optional, can leave blank.
%		Recommended format: {Instructor: Jane Doe}
\newcommand{\Instructor}{}
%	Course: optional, can leave blank.
%		Recommended format: {Course: Biostatistics 101}
\newcommand{\Course}{}



%	******	Document body begins here	**********************

\begin{document}

%	Title page
\begin{frame}[plain]
	\titlepage
\end{frame}

%	******	Everything through the above line must be placed at
%		the top of any TeX file using the statsTeachR standard
%		beamer preamble. 
\begin{frame}{Overview}
\begin{itemize}
\item Parallel computing allows us (for certain problems) to split a big job up into many smaller parts and run them in parallel.  
\item Our laptops or desktops can split a job up into as many processors as are available.  
\item This is likely a relatively small number like 4, 8, or 12. 
 \end{itemize}
\end{frame}

\begin{frame}{Overview}
\begin{itemize}
\item {\bf Cluster computing} connects many computers (nodes) together on a local network. 
\item This allows a pooling of resources to increase computing power.
\item This allows a user to access hundreds or even {\it thousands} of cores (if they need them). 
\end{itemize}
\end{frame}

\begin{frame}{MGHPCC}
\begin{itemize}
\item The Massachusetts Green High Performance Copmputing Cluster (MGHPCC) is one of these clusters.  
\item Each of the participating instiutions has their own distinct cluster. We will be using the UMass cluster.
\item In order to connect to this cluster, a user has to be on the UMass campus (or use VPN to the UMass campus.)
\item We will need to VPN into the UMass network since we are currently off campus.  
\end{itemize}
\end{frame}

\begin{frame}{VPN information}
\begin{itemize}
\item Server address: vpn2.oit.umass.edu 
\item Group name: umass
\item Password: vpn4umass
\item Account name: bip2014 (temporary for BIP2014)
\item Password:  (temporary for BIP2014)
\end{itemize}
\end{frame}

\begin{frame}{Accessing the MGHPCC}
\begin{itemize}
\item Register for the MGHPCC (for UMass): https://www.umassrc.org/hpc/
\item A login and password will be assigned to you. 
\item This will be different than your UMass IDor email address.
\end{itemize}
\end{frame}


\begin{frame}{Accessing the MGHPCC}
\begin{itemize}
\item Log in with: ssh username@ghpcc06.umassrc.org
\item Enter password when prompted.  
\item You are now connected to the cluster!
\item Remember: You can only access the MGHPCC if you are on UMass's network (physically on campus or VPN)
\end{itemize}
\end{frame}

\begin{frame}{Transferring Data to MGHPCC}
\begin{itemize}
\item Easy ways to do this are with: scp or sftp
\item You can also use rsync, but this is a bit more complicated.
\end{itemize}
\end{frame}


\begin{frame}{Transferring Data to MGHPCC: sftp}
\begin{itemize}
\item Connect to the MGHPCC using: sftp username@ghpcc06.umassrc.org and enter your password.  
\item You can not transfer files to the server from your local computer and vice versa.  
\item The command {\bf put} pulls a file from your local drive to the server.
\item The command {\bf get} pulls a file from your the server to your local drive.
\end{itemize}
{\tt \$ put myData.csv /home/gjm43a/BIP2014}\\
{\tt \$ get results.csv ~/clusterResults}
\end{frame}

\begin{frame}{Transferring Data to MGHPCC: scp}
\begin{itemize}
\item We can also use the scp command.
\item This will transfer our files from our local machine to the server. 
\item Usage is scp [source] [destination]
\item Running this command will prompt you for your password and then the file will be transferred.  
\item scp is better than sftp for transferring directories.  
\end{itemize}
{\tt \$ scp myData.csv gjm43a@ghpcc06.umassrc.org:/home/gjm43a}\\
{\tt \$ scp gjm43a@ghpcc06.umassrc.org:/home/gjm43a myData.csv}
\end{frame}

\begin{frame}{Transferring Data to MGHPCC: scp}
\begin{itemize}
\item With scp we can also transfer entire folders.
\item To do so we use the -r option
\end{itemize}
{\tt \$ scp -r /Users/gregorymatthews/Dropbox/BiPSandbox/code gjm43a@ghpcc06.umassrc.org:/home/gjm43a}\\
{\tt \$ scp -r gjm43a@ghpcc06.umassrc.org:/home/gjm43a/code /Users/gregorymatthews/Dropbox/BiPSandbox}
\end{frame}



\begin{frame}{Software}
\begin{itemize}
\item Once logged in, you will be placed in your home directory (/home/username). 
\item Full list of available software: http://wiki.umassrc.org/wiki/index.php/Provided\_Software  
\item There is a wide array of available software options. 
\end{itemize}
\end{frame}

\begin{frame}{Software}
\begin{itemize}
\item In order to use software on the server, you'll need to load modules that contain the software.  
\item We are interested in using R here.  
\item To load R we use the command: module load R/3.0.1
\item We can also unload R with the command:  module unload R
\end{itemize}
\end{frame}

\begin{frame}{Software}
\begin{itemize}
\item Loading the R module will allow us to run R interactively on the MGHPCC servers.  
\item But this isn't really the best use of the cluster.  
\end{itemize}
\end{frame}

\begin{frame}{Batch Mode}
\begin{itemize}
\item Rather than running R code interactively, we can also run R in BATCH mode.  
\item At the command prompt, we submit a job by typing: R CMD BATCH [options] filename.R
\item But we don't really want to do this either.  
\item We want to submit batch jobs to the cluster.  
\end{itemize}
\end{frame}

\begin{frame}{Submitting jobs to the cluster}
\begin{itemize}
\item In order to submit a job to the cluster we need: 
\begin{itemize}
\item A R script that we wish to run
\item A shell script that calls the R script and submits the job to the cluster.  
\end{itemize}
\item We will then submit the job to the LSF (Load Sharing Facility) scheduler.  
\item {\bf Note}: Commands submitted to LSF are just like if they were run from a command prompt. 
\end{itemize}
\end{frame}


\begin{frame}{LSF common commands}
\begin{itemize}
\item bsub - submit a job
\item bkill - kill a job
\item bjobs - view status of jobs
\item bpeek - view output / error files
\item bhist - job history
\item bqueues - available queues
\end{itemize}
\end{frame}







\begin{frame}{Batch Mode}
\begin{itemize}
\item Once we have a .R file that we want to run we can submit it to the LSF job scheduler.  
\item What actually gets submited to the LSF scheduler is a shell script that calls the R file (which we'll call example.R). 
\item We need to create a file that contains all of the code we would have run at the shell prompt.  
\item Example of a shell script is below. We'll call this shell script example.sh
\end{itemize}
{\tt \$ module load R/3.0.1 }\\
{\tt \$ R CMD BATCH --vanilla example.R}\\
\end{frame}

\begin{frame}{Submiting jobs to the cluster}
\begin{itemize}
\item The command {\bf bsub} allows us to submit the shell script to the cluster. 
\item The command below will submit the shell script example.sh to the cluster.  
\end{itemize}
{\tt \$ bsub example.sh}
\end{frame}


\begin{frame}{Submiting jobs to the cluster: Options}
\begin{itemize}
\item We can also set many options in our job submission. 
\item -n: Number of cores requested
\item -W: Wall clock time
\item -R: Memory per job
\item -q: Which queue to submit to
\end{itemize}
{\tt \$ bsub -n 4 -R "rusage[mem=2048]" -W 0:10 -q long example.sh}
\end{frame}

\begin{frame}{Submiting jobs to the cluster: Local}
\begin{itemize}
\item An example shell script is below. 
\item Here we submit to the parallel queue.
\item Commands to submit is same as before: bsub < example.sh
\end{itemize}
{\tt \#!/bin/bash\\
\#BSUB -R rusage[mem=2048] \# ask for memory\\
\#BSUB -R span[hosts=1]    \# all the cores on one machine\\
\#BSUB -W 0:10             \# not sure what this is doing\\
\#BSUB -q short  \# which queue we want to run in\\

module load R/3.0.1\\
R CMD BATCH /home/gjm43a/BiP/localParallelPermTest.R}
\end{frame}

\begin{frame}{Submiting jobs to the cluster: Distributed}
\begin{itemize}
\item We can also specify options in the shell script that we submit to the LSF scheduler.  
\item An example shell script is below. 
\item Commands to submit: bsub < example.sh
\end{itemize}
{\tt \#!/bin/bash\\
\#BSUB -n 100\\
\#BSUB -R rusage[mem=2048] \# ask for memory\\
\#BSUB -W 0:10\\
\#BSUB -q {\bf parallel}  \# which queue we want to run in

module load R/3.0.1\\
R CMD BATCH /home/gjm43a/BiP/distrParallelPermTest.R /home/gjm43a/BiP/file1.Rout\\
R CMD BATCH /home/gjm43a/BiP/distrParallelPermTest.R /home/gjm43a/BiP/file2.Rout\\
R CMD BATCH /home/gjm43a/BiP/distrParallelPermTest.R /home/gjm43a/BiP/file3.Rout\\
R CMD BATCH /home/gjm43a/BiP/distrParallelPermTest.R /home/gjm43a/BiP/file4.Rout}
\end{frame}



\end{document}

